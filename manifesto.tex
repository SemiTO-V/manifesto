\documentclass{article}
\usepackage{graphicx} % Required for inserting images

\title{SemiTO-V Manifesto}
\author{Tan Siret Akıncı}
\date

\begin{document}
\maketitle
\begin{center}
    \includegraphics[scale=0.7]{semito-v_logo.png}
\end{center}
\maketitle

\section{Vision}

SemiTO-V envisions a future where computing is truly democratized—a future in which both software and hardware are fully open-source, free from black boxes, proprietary restrictions, and vendor lock-in. This future is made possible by RISC-V, the open ISA (Instruction Set Architecture), which, thanks to huge industry support, unlocks unprecedented innovation, collaboration, and accessibility in computing. As students, we aim to participate in this computing revolution. In line with this vision, we want to build our careers in RISC-V-related fields and contribute to RISC-V as undergraduates and graduates alike. To achieve this, we believe the best way to start is to form a student team dedicated to RISC-V ISA and technologies around it.


\section{Mission}
SemiTO-V is a student team using RISC-V ISA based cores to create custom processors, firmware, and software. We’re mastering everything from RTL design to low-level software development for RISC-V platforms (RV32 and potentially RV64). Our mission is to use, leverage and enhance standardized/popular RISC-V cores, making our own processor designs (MCU, PPU etc), optimizing them for real-world applications (IMU, AI accelerator etc) while improving their software/firmware ecosystem. Biggest problem of existing RISC-V hardware is lack of good firmware and software compatibility. For that reason, we want to focus on maximizing performance, power efficiency, and software support for existing RISC-V chips, ensuring they are fully utilized across embedded systems, cloud, edge computing, HPC and beyond.


\subsection{What We Do}

\paragraph{Hardware and Logic Dev:}

\begin{itemize}
    \item Use cores based on RISC-V ISA (RV32/RV64), add custom instructions and modules
    \item Implement SoCs targeting FPGAs
    \item Optimize microarchitectures for specific workloads (ML, cryptography, DSP)
    \item Interface with peripherals, make our processors wifi-enabled.
    \item Explore advanced features such as pipelining and parallel processing (potentially for HPC)
\end{itemize}
\paragraph{Software and Instruction Dev:}

\begin{itemize}
    \item Develop/improve bare-metal firmware, RTOS (FreeRTOS) and time-sharing OS (Linux for RV64) implementations
    \item Port and optimize compilers for our processors
    \item Contribute to the software that are important for RISC-V ecosystem (Linux kernel, LLVM, vector libraries)
    \item Create our own SDKs, toolchains and simulators for our designs
\end{itemize}

\subsection{Our Approach}

\paragraph{Open Source All The Way:}

\begin{itemize}
    \item All our designs, code, and documentation live in public repos
    \item We build on and contribute back to open source RISC-V projects
    \item We document everything
\end{itemize}

\paragraph{Practical Learning:}

\begin{itemize}
    \item Learn by doing: Design, code, test, iterate
    \item Learn by following and using newest RISC-V based technologies
    \item Apply classroom theory by using FPGA boards, building upon open processor architectures and using low-level programming languages for firmware/software development
\end{itemize}

\paragraph{Community Focus:}

\begin{itemize}
    \item Share knowledge through workshops and tutorials
    \item Collaborate with other RISC-V groups globally
    \item Bridge academia, userspace and industry through open source philosophy
\end{itemize}

\paragraph{Sector Focus:}

\begin{itemize}
    \item Get to know the chip and RISC-V industries
    \item Beta test latest open source RISC-V products and contribute to their development (RISC-V Development Partner Program, Codasip Studio Uni Program, Deepcomputing Framework ambassadorship)
    \item Share our projects with companies for evaluation
    \item Discover internship and job opportunities about RISC-V thanks to team projects
\end{itemize}

\section{Join Us}

If you're passionate about processor architectures, hardware description languages, compilers, or low-level software—come build with us! From RTL to riscv assembly, we're hacking the full stack!

\paragraph{Send applications to: semitofive@gmail.com}

\subparagraph{\textit{SemiTO-V: RISC-V Student Team of Politecnico di Torino}}
\end{document}

